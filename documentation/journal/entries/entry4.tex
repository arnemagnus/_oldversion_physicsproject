\section{Week 41}

When using \textsc{NumPy} or \textsc{SciPy} interpolation, e.g.\ by use of
\textsc{RectBivariateSpline}, the indexing is not what we expect from Python. 
This is due to how the \textsc{NumPy} meshgrids share indexing pattern with 
\textsc{Matlab}. In particular, this means that, when generating the 
interpolation object, the correct call structure is as follows:

\begin{code}
    \captionof{listing}{The correct way to generate a \textsc{RectBivariateSpline}}
    \label{code:correctspline1}
    \begin{minted}{python}
    
    from scipy.interpolate import RectBivariateSpline as RBSpline
    import numpy as np

    # ----------------------------------------------------- #
    # Function and variables defined somewhere in the above # 
    # ----------------------------------------------------- #

    x = np.linspace(dx/2, Nx-dx/2, Nx)
    y = np.linspace(dy/2, Ny-dy/2, Ny)

    z = foo(x, y)

    spline = RBSpline(y, x, z)
\end{minted}
\end{code}

rather than

\begin{code}
    \captionof{listing}{An incorrect way of generating a \textsc{RectBivariateSpline}}
    \label{code:incorrectspline}
\begin{minted}{python}
    spline = RBSpline(x, y, z) # <-- Incorrect
\end{minted}
\end{code}

Similarly, the correct call structure when evaluating the interpolated object
is
\begin{code}
    \captionof{listing}{The correct way to evaluate the interpolated object}
    \label{code:correctsplineevaluation}
\begin{minted}{python}
    x2 = np.linspace(x_min, x_max, NI)
    y2 = np.linspace(y_min, y_max, NJ)

    z2 = spline.ev(y2, x2)
\end{minted}
\end{code}

rather than

\begin{code}
    \captionof{listing}{An incorrect way of evaluating the interpolated object}
    \label{code:incorrectsplineevaluation}
\begin{minted}{python}
    z2 = spline.ev(x2, y2) # <-- Incorrect
\end{minted}
\end{code}



