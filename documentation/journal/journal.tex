\documentclass[a4paper,12pt]{article}

% For å laste inn fonter:
\usepackage{fontspec}
% Diverse formatering, deriblant (men ikke begrenset til) URL'er
\usepackage{xunicode,xltxtra,url,parskip}

\RequirePackage{color,graphicx}
\usepackage[usenames,dvipsnames]{xcolor}
% Bedre A4-formatering:
\usepackage[big]{layaureo}

% Oppsett av hyperref-pakken
\usepackage{hyperref}
\definecolor{linkcolour}{rgb}{0,0.2,0.6}
\hypersetup{colorlinks,breaklinks,urlcolor=linkcolour,
  linkcolor=linkcolour}

% Runde sitatparenteser med \citep
\usepackage[round]{natbib}

% For å bl.a. kunne loope over filnavn:
\usepackage{tikz}

%--------------------------------------------------%
% Matematiske funksjoner, og egendefinerte makroer %
% -------------------------------------------------%

% "Go-to"-pakken for matematikk i LaTeX:
\usepackage{amsmath}

% Matematiske symboler med fet skrift
\usepackage{bm}

\newcommand{\vect}[1]{{\bm{\mathrm{#1}}}}
%---------------------------BEGIN DOCUMENT---------------------------%

\begin{document}


%-------------------------------TITLE--------------------------------%

\par{\centering
    {\LARGE TFY4510 \textsc{Specialization Project in Physics}
    }\medskip\par}

\par{\centering
    {\large Journal, courtesy of Arne Magnus Tveita Løken
    }\bigskip\par}

%----------------------------MAIN CONTENT-----------------------------%

% Organiserer journalføringen i enkeltfiler i underkatalogen
% 'entries', og laster filene sekvensielt.
%
% OBS: Husk å oppdatere antallet inputfiler før kompilering for 
% å få med alle journalføringene: 

\foreach \c in {1, ..., 1} {\input{entries/entry\c.tex}}
  
%------------------------------BACKMATTER-----------------------------%
\bigskip

\bibliographystyle{apalike}
\bibliography{mybibliography}

\end{document}


% Local Variables:
% TeX-engine: luatex
% TeX-master: t
% End:

